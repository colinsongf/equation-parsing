  Our work is related to two main ares of research : semantic parsing and
  automatic math word problem solving.

  \noindent
  \textbf{Automatic Math Word Problem Solvers} Mathematical relation
  extraction is an essential component of math word problem
  solvers. However, most problem solvers restrict the space of
  equations that can be generated, either by using equation templates
  or making assumptions on the problem type. \cite{KushmanZeBa14} solves
  simultaneous equation math word problems, by inducing equation
  templates and aligning numbers mentioned in the problem to template
  slots. Unlike ours, their system does not ground variables and
  cannot handle compositional expressions. Also, our system explores
  the entire space of equations, and does not assume that the equation
  templates for the test data has been seen in the training examples.

  There has been approaches to automatic math problem
  solving \cite{RoyViRo15,HosseiniHaEt14}, which focus on elementary
  school problems, and hence has a few number of possible output
  equations. \cite{RoyViRo15} focuses on questions which can be
  answered by choosing two numbers from the problem, and applying one
  of four basic operations on it. \cite{HosseiniHaEt14} looks at
  addition and subtraction problems, which they model by state
  transitions. Both these approaches searches a small space of
  equations, and can handle only simple syntax.

  \noindent \textbf{Semantic Parsing} There has been a lot of work in
  mapping natural language text to formal meaning
  representation. Some \cite{ZettlemoyerCo05,GeMo06} assume access to
  manually annotated formal representation for every sentence, whereas
  others \cite{ECGR09,KateMo07,GoldwasserRo11,GRCR11} reduce the
  amount of supervision using domain knowledge. Reinforcement learning
  has also been used \cite{BCZB09,BranavanZeBa10,VogelJu10} to
  automatically understand and execute instructions. All the above
  approaches depend on a CCG-like parse of the entire sentence, and
  restrict their domain to simple short sentences. In constrast, our
  equation parse is only constructed on quantities mentioned in text,
  and token spans which represent variable entities for the output
  equation. This allows us to efficiently parse longer compositional
  sentences.
