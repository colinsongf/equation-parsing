  Understanding text often involves reasoning with respect to
  quantities mentioned in it. Understanding a news article statement
  that {\em``Emanuel's campaign contributions total three times those
  of his opponents put together''} requires identifying relevant
  entities, the mathematical relations expressed among them in text,
  and how to compose them. Similarly, an elementary school math word
  problem such as: {\em``The difference of twice a number and 3 is
  five more than thrice another number.''} requires realizing that we
  deal with the relations between two numbers, knowing the meaning of
  {\em``difference''} and composing the right equation --
  {\em``five''} needs to be added only after “three” is multiplied
  with the number.

  The first step in this process involves identifying relevant
  variables and {\em grounding} them in text, that is, detecting a
  span of text which describes what the variable represents. Grounding
  is necessary to support reasoning, as well as analysis across
  multiple sentences. However, the text often does not mention
  {\em``variables''} explicitly. For example, the sentence
  {\em``Flying with the wind , a bird was able to make 150 kilometers
  per hour.''}  describes a mathematical relation between the speed of
  bird and the speed of wind, without mentioning ``speed''
  explicitly. There is also ambiguity regarding mention choice while
  grounding variables. For example, in the following sentence-equation
  pair

  \setlength{\tabcolsep}{6pt}
  \begin{table}[H]
   \centering \small
   \begin{tabular}{|lp{5cm}|}
     \hline Example 1 &  \\
     \hline Sentence & {\em City Rentals rent an intermediate-size car
  for 18.95 dollars plus 0.21 per mile.} \\
     \hline Equation & $V_1=18.95+0.21*V_2$ \\
     \hline 
   \end{tabular}
   \label{tab:example1}
  \end{table}
  
  \noindent the variable $V_1$ could be grounded to {\em``City
  Rentals''}, {\em``rent''}, or {\em``intermediate-size car''}. The
  later part of the text might mention another rental company, making
  {\em``City Rentals''} the prime entity of this sentence. Again, the
  text might go on to discuss {\em``City Rental full size''} cars,
  making {\em``intermediate cars''} more important.
  
  In this paper, we introduce the {\em Equation Parsing} task: given a
  sentence describing a mathematical relation between entities, the
  goal is to map it to an equation representing the relation, and to
  determine what the variables the text refers to. We propose a novel
  approach to address the problem of equation parsing, involving the
  identification and grounding of variables and predicting the
  corresponding equation. First, we detect and normalize all
  quantities in the sentence, and prune the irrelevant quantities,
  which do not take part in the relation. Next we develop a structured
  prediction model to jointly predict variable grounding and the
  corresponding equation.  As seen in example 2, there are often
  multiple spans of text which can be valid grounding of a
  variable. Here, $V_1$ can be correctly grounded to both {\em``the
  first one''} or {\em``both numbers''}. Since exact segmentation of
  mentions can be ambiguous, we ground variables to tokens in the
  sentence, which are either nouns, verbs or adjectives. A grounding
  is said to be valid if the system grounds to a token within a gold
  grounded span.  Since there can be multiple valid groundings for a
  problem, we model the best grounding to be latent, and let the model
  figure out which grounding is best for it, to generate equations.
  
  
  \setlength{\tabcolsep}{6pt} 
  \begin{table} 
         \centering 
         \small 
         \begin{tabular}{|lp{5cm}|} \hline
         Example 2 & \\ \hline 
         Sentence & {\em If 10 is added to two numbers,
         the first one will be 5 more than thrice the second.} \\ \hline
         Equation & $V_1+10=5+3*(V_2+10)$\\ \hline 
         Grounding & $V_1= \{\text{``the first one'', ``two numbers''}\}$
         $V_2= \{\text{``the second'', ``two numbers''}\}$\\ 
        \hline 
        \end{tabular} 
        \label{tab:example2} 
  \end{table}
  
  
  Our inference algorithm relies heavily on the idea of an equation
  parse of a sentence. The parse can effectively capture
  compositionality, can directly generate the necessary equation from
  text, and it provides a way to explore the space of all equations
  efficiently.

  We develop and annotate datasets \footnote{Our datasets will be made
  available with the final version.} for evaluation and show that our
  method can handle the equation generation and grounding task quite
  well.

  The next section presents some related work on mathematical
  reasoning and semantic parsing. We then present our
  equation parse representation, describe our algorithm to generate
  the parse from text, and conclude with experimental evaluation.
  
